% ========================================================
% PROJECT TEMPLATE, *KNOWLEDGE DISCOVERY*, Summer 2018
% University of Potsdam, by Christoph Schommer, University
% of Luxembourg; 
% ========================================================

\documentclass[11pt]{article}
\usepackage{geometry}
\usepackage{enumitem}
\usepackage{graphicx}
\usepackage[usenames,dvipsnames]{color}

%
\def\MakeMeBlue#1{\textcolor{Blue}{#1}}
\pagenumbering{arabic}
%

\parindent 0pt

%

\title{\MakeMeBlue{Market Basket Analyse on Online Retail}}
\author{Uchralt Temuulen}
\date{2018 07 08}

% ========================================================
\begin{document}
\maketitle

% ========================================================
\section{Introduction}
The motivation in this task is to know the customer behaviour on online retail for merchandise products. 
Also how do different group of customer that spend different 
amount of money behave. What kind of products are they most freuently buying? Are there differences in these groups? What kind of products are connected to each outher mostly? 
We used for these the market basket analyse method. We found out that there our not big differences in different customers groups. They all buy allmost the same article. 

ss


 
\\

An example of using bullet points, in case it is needed.
\begin{itemize}[leftmargin=1cm]
   \item Text 1
   \item Text 2
\end{itemize}

% ========================================================
\section{Related Work}
An example of how to insert references in the article~\cite{hu2011}.


% ========================================================
\section{Methodology}
Introduce the machine learning technique(s) or the algorithm(s) used in the project. \\

An example of inserting figures. Position and width of the figure can be adjusted as needed.
\begin{figure}[!htp]        
  \centering
    \includegraphics[width=0.2\textwidth]{image.jpg}
    \caption{Brief description of the figure.}
\end{figure}

% ========================================================
\section{Implementation}

Introduce the data used in the experiments, the setup of the experiments, and the results and/or comparisons. \\

An example of inserting tables. Position of the table can be adjusted as needed.
\begin{table}[!htp] 
  \centering
    \begin{tabular}{| l c r |}   % Alignments can be set here
    \hline
    1 & 2 & 3 \\
    4 & 5 & 6 \\ \hline
    7 & 8 & 9 \\
    \hline
    \end{tabular}
  \caption{Brief description of the table.}
\end{table}

% ========================================================
\section{Conclusion}
Conclude the whole project in short text.

% ========================================================

\begin{thebibliography}{9}

\bibitem{hu2011}
  Hu, Yi and Li, Wenjie,
  \textit{Document sentiment classification by exploring description model of topical terms},
  Computer Speech \& Language,
  25-2, pages 386--403,
  Elsevier, 2011.

\end{thebibliography}
 
\end{document}